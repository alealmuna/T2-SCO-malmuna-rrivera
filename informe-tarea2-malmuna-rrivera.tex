\documentclass[letter, 10pt]{article}
\usepackage[spanish]{babel}
\usepackage[utf8]{inputenc}
\usepackage{amsfonts}
\usepackage{amsmath}
\usepackage{graphicx}
\usepackage{url}
\usepackage[top=3cm,bottom=3cm,left=3.5cm,right=3.5cm,footskip=1.5cm,headheight=1.5cm,headsep=.5cm,textheight=3cm]{geometry}


\begin{document}
\title{Módulo Redes avanzadas \\ \begin{Large}Programación Distribuida y Criptografía\end{Large}}
\author{Manuel Almuna (ale@amucan.com) - Renato Rivera (renato@amucan.com)}
\date{\today}
\maketitle

\begin{abstract}
Informe sobre la experiencia de programación distribuida y 	
los problemas de implementación asociados al desarrollo de un servicio en red y	el	
uso de técnicas criptográficas.

\end{abstract}
\section{Especificaciones de software}
Versión Java: 
\begin{itemize}
 \item 1.7.01
 \item OpenJDK Runtime Environment (IcedTea 2.6.6)
 \item OpenJDK 64-Bit Server VM
\end{itemize}
Paquetes:
\begin{itemize}
 \item java.net
 \item java.io
 \item java.util
 \item javax.crypto
 \item java.security
\end{itemize}


\section{Diseño de la solución}
Se implementa una aplicación Cliente-Servidor basada en sockets TCP. Esta permite enviar un mensaje de forma
segura comparando los MAC calculados en ambos nodos de la conexión.
\newline
\\
Para el Servidor se crea un socket que queda escuchando en el puerto indicado.
\begin{verbatim}
 17.  sersock = new ServerSocket(PORT);
\end{verbatim}
Al recibir una petición, se abre un socket para atender la conexión.  
\begin{verbatim}
 21. sock = sersock.accept();
\end{verbatim}
Luego se recibe el mensaje.
\begin{verbatim}
 24. BufferedReader is = new BufferedReader(
 25.        new InputStreamReader(sock.getInputStream()));
 27. String msg = is.readLine();
\end{verbatim}
Se calcula el MAC del mensaje y reenvía al cliente
\begin{verbatim}
 28. String mac = HMAC.hmacDigest(msg, cryptoKey, "HmacMD5"); 
 30. PrintStream ios = new PrintStream(sock.getOutputStream());
 31. ios.println(mac);
\end{verbatim}

El cliente se conecta a la IP y puerto indicados.
\begin{verbatim}
 41. sock= new Socket(ip, Server.PORT);
\end{verbatim}
Se envía el mensaje dado por el usuario, utilizando Scanner de java.util.
\begin{verbatim}
 48. String msg = scanner.next(); 
 49. ps.println(msg); 
\end{verbatim}
Se recibe la MAC calculada por el servidor.
\begin{verbatim}
 50. String recv_mac = is.readLine();  
\end{verbatim}
Se comparan las MAC a través de la función checkMsg, la cual calcula la MAC de un mensaje
y lo compara con un MAC dado, en este caso el del servidor.
\begin{verbatim}
 11. String mac = HMAC.hmacDigest(msg, cryptoKey, "HmacMD5");
 14. if(mac.equals(h)){                                 
 15.    System.out.println("Message received safely.");
\end{verbatim}

\newpage
\section{Conclusiones}
El uso de HMAC permite una comunicación segura, que dependerá del tamaño de salida del hash, y en la calidad
y tamaño de la llave. También de la potencia criptográfica de la función de hash elegida.


\section{Referencias}


\end{document} 
